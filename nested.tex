\documentclass{acmtog}

\usepackage[usenames]{xcolor}

\newcommand{\leo}[1]{{\bf\textcolor[rgb]{0.9,0.0,0.0}{Leo: #1}}}
\newcommand{\todo}[1]{{\bf\textcolor[rgb]{0.9,0.0,0.0}{TODO: #1}}}

\acmVolume{VV}
\acmNumber{N}
\acmYear{YYYY}
\acmMonth{Month}
\acmArticleNum{XXX}  
\acmdoi{10.1145/XXXXXXX.YYYYYYY}

%\acmVolume{28}
%\acmNumber{4}
%\acmYear{2009}
%\acmMonth{August}
%\acmArticleNum{106}  
%\acmdoi{10.1145/1559755.1559763}

\begin{document}

\markboth{L. Sacht, A. Jacobson and E. Vouga}{Nested Exterior Cages via Reverse Flow with Collision Response}

\title{Nested Exterior Cages via Reverse Flow \\ with Collision Response} % title

\author{LEONARDO SACHT
\affil{Universidade Federal de Santa Catarina {\upshape and} Instituto Nacional de Matematica Pura e Aplicada}
ALEC JACOBSON
\affil{\leo{please put names of universities}}
\and 
ETIENNE VOUGA
\affil{\leo{please put names of universities}}}

\category{\leo{Select these correctly} I.3.7}{Computer Graphics}{Three-Dimensional Graphics and Realism}[Animation]
\category{I.3.5}{Computer Graphics}{Computational Geometry and Object Modeling}[Physically based modeling]

\terms{\leo{Select these correctly} Experimentation, Human Factors}

\keywords{\leo{Select these correctly} Face animation, image-based modelling, iris animation, photorealism, physiologically-based modelling}

%\acmformat{Pamplona, V. F., Oliveira, M. M., and Baranoski, G. V. G. 2009. Photorealistic models for pupil light
%reflex and iridal pattern deformation.  {ACM Trans. Graph.} 28, 4, Article 106 (August 2009), 11 pages.\newline  DOI $=$
%10.1145/1559755.1559763\newline http://doi.acm.org/10.1145/1559755.1559763}

\maketitle

%\begin{bottomstuff} 
%Manuel M. Oliveira acknowledges a CNPq-Brazil fellowship (305613/2007-3). Gladimir V. G. Baranoski %acknowledges a
%NSERC-Canada grant (238337). Microsoft Brazil provided additional support.

%Authors' addresses: land and/or email addresses.
%\end{bottomstuff}


\begin{abstract} 
Many tasks in geometry processing and physical simulation benefit from multiresolution hierarchies. The benefits are particularly striking for tasks that require solving Eulerian PDEs over the volume of an object, where downsampling the object cubically reduces the size of the problem. The coarse discretization must satisfy several desiderata, in order for the multiresolution algorithm to be efficient and accurate: the coarsened object should respect the topology of the original, it should \emph{enclose} the original object, yet should fit the original as tightly as possible, i.e. include a minimal amount of extraneous volume. Existing techniques for building a multiresolution hierarchy, such as fitting a regular grid around the object, voxelizing the object, or meshing the object followed by decimation, violate one or more of these desiderata. We propose a solution that satisfies all three requirements: we first mesh the object and decimate its surface to get a sequence of (possibly non-nesting) coarsened surfaces. We then move the surface of the second coarsest layer along a flow designed to place it inside coarsest layer. We repeat this until the original surface is inside all coarser layers. Finally, we inflate the original surface back to its original shape, responding to any collisions that occur with the coarser layers that might occur during inflation. From coarse to fine, each layer then fully encloses the next while retaining a snug fit and respecting the original surface topology. We show how these hierarchies are useful for fast PDE solving, low-frequency data upsampling, and geometric modeling.
\end{abstract}



\section{Introduction}
\label{sec:introduction}

As the complexity and size of objects we try to simulate and model continues to grow, acceleration algorithms for handling these large meshes are becoming increasingly important. One powerful technique is to decompose the high-resolution mesh into a hierarchy of coarser approximations or \emph{cages}: for example, Eulerian PDEs can be solved efficiently on the fine mesh using any of several successful multiresolution FEM techniques, which first solve for a coarse approximate solution on the coarsest levels of the hierarchy, and then polishing that solution on successively finer levels. Coarse cages also find heavy use in animation, where artist adjustments of the coarse cage is propagated to the fine model, and in collision detection, where a coarse cage can be used to efficiently but conservatively exclude pairs of objects from having collided. For all these applications, the key to good performance is the ability to generate a good-quality multiresolution hierarchy.

\noindent \textbf{Desiderata} Although different applications place different emphases on what makes for a desirable multiresolution hierarchy, an ideal coarse cage possesses three properties that are particularly important across a wide spectrum of applications:
\begin{itemize}
\item It completely \emph{encloses} the fine mesh as well as any of the finer levels of the multiresolution hierarchy. Most algorithms for transferring the pose of a cage to a detailed object only guarantee small distortion of the object and lack of artifacts like element inversion if the object is entirely contained withing the cage, for instances. And if the cage is to be used to accelerate collision detection, it is essential that the cage conservatively enclose the object. \todo{Some plausible FEM motivation for this item}
\item It respects the \emph{topology} of the fine mesh and does not introduce any large global errors in the object geometry. It is important to differentiate between two types of distances when building the hierarchy: \emph{embedded distance} in ambient space, and \emph{intrinsic distance} between two points, the distance of the shortest path through the material connecting the two points. It is possible for two points that are nearby in embedded distance to be very far apart in intrinsic distance: consider, for instance, a model of a human leaning over and nearly touching her toes: the fingertips and toes are near in ambient space but intrinsically are meters apart. It is critical for many applications that the coarsening process does not fuse together these parts of an object. In the case of the human model, it is unacceptable if trying to animate the model's arm causes the leg to move as well. As another example, consider calculating the vibration modes of a broken ring; if the coarsening process fuses the ring together, the modes computed will be totally incorrect.
\item It fits \emph{tightly} around the next-coarser level of the hierarchy. Formally, tightness can be quantified using a user-specified energy that minimizes Hausdorff distance between the cage and the fine mesh, or minimizes the volume enclosed by the two meshes, etc. A tight fit minimizes the error incurred by simplifying the input mesh.
\end{itemize}
Existing popular approaches to building a coarse cage satisfy some, but not all, of these three properties. For instance,
\begin{itemize}
\item Simple decimation of the original surface mesh yields cages that fit very tightly and preserve a high amount of detail for a given coarsened resolution, and can preserve the topology of the original mesh, but there is no guarantee that the decimated mesh will enclose the original.
\item Voxellizing the object to a regular grid is guaranteed to produce an enclosing cage, but the fit will not be tight and intrinsically distant features can be fused together if the grid is too coarse. Using an adaptive grid or octree can improve the fit and topological fidelity, but unless the object contains mostly axis-aligned features the resulting cage will not efficiently fit the object.
\item Level-set methods can find a well-fitting cage that encloses the object, but can fuse spatially proximate features, changing the object topology.
\end{itemize}

\noindent\textbf{Our Contribution} We propose an algorithm for computing a multiresolution hierarchy of cages that unlike previous approaches is \emph{guaranteed} to satisfy all three of the desired properties. Given the mesh $M_i$ for level $i$ of the hierarchy, we compute the next-coarsest level $M_{i+1}$ using three steps (see figure XXX): first, we perform topology-preserving decimation to get a coarse, but not enclosing, cage $C_{i+1}$. We then shrink $M_i$, using a mixture of area- and distance-minimizing flow (see section XXX) until $M_i$ is completely enclosed by $C_{i+1}$. Finally, we linearly expand the vertices of $M_i$ back towards their original positions. We perform this expansion as a physical simulation (see section XXX), where the motion of $M_i$ is prescribed, contact response guarantees that $C_{i+1}$ remains an enclosing cage, and a fairness energy maximizes the tightness of the fit. The final position of $C_{i+1}$ after this simulation becomes the next cage $M_{i+1}$.

\section{Related work}
\label{sec:related}

\section{Method}
\label{sec:method}

\section{Implementation}
\label{sec:implementation}

\section{Results and applications}
\label{sec:results}

\section{Conclusion}
\label{sec:colnclusion}

\begin{acks}
\end{acks}

\bibliographystyle{acmtog}
\bibliography{nested}


\received{September 2008}{March 2009}

\end{document}

\section*{Appendix: Flow counterexample}
%
\label{app:counterexample}
%
There exist polyhedra that cannot be continuously flowed inside themselves~\cite{Mathoverflow:206750}.
%
Consider the surface constructed in \reffig{vouga-fan}, right. Though inward
pointing normals can be defined at all vertices, any non-negative (even
infinitesimal) inward movement of the vertices will introduce ``edge-edge''
collisions with the original surface.
%
To see this, consider the set of points inside the polyhedron that can be "seen" by the central vertex $c$ of the star. This set changes discontinuously for any motion of $c$: no matter which direction is chosen to flow $c$, there exists some neighbor $n$ such that $c$ loses sight of not only $n$, but also an entire neighborhood around $n$. Therefore there exists no inward flow of all of the vertices where $c$ maintains sight with all of its neighbors.
\begin{figure}[hb]
\includegraphics[width=\linewidth]{figs/vouga-fan}
\caption{Take every other vertex around the star surface on the left and
rotate it about the origin to create the ``origami pinwheel'' on the right.
Vertices of this new surface cannot be continuously flowed inward without
creating intersections with the original surface.}
\label{fig:vouga-fan}
\end{figure}

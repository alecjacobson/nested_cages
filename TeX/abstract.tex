\begin{abstract} 
Many tasks in geometry processing and physical simulation benefit from
multiresolution hierarchies. The benefits are particularly striking for tasks
that require solving Eulerian PDEs over the volume of an object, where
downsampling the object cubically reduces the size of the problem. The coarse
discretization must satisfy several desiderata, in order for the
multiresolution algorithm to be efficient and accurate: the coarsened object
should respect the topology of the original, it should \emph{enclose} the
original object, yet should fit the original as tightly as possible, i.e.
include a minimal amount of extraneous volume. Existing techniques for building
a multiresolution hierarchy, such as fitting a regular grid around the object,
voxelizing the object, or meshing the object followed by decimation, violate
one or more of these desiderata. We propose a solution that satisfies all three
requirements: we first mesh the object and decimate its surface to get a
sequence of (possibly non-nesting) coarsened surfaces. We then move the surface
of the second coarsest layer along a flow designed to place it inside coarsest
layer. We repeat this until the original surface is inside all coarser layers.
Finally, we inflate the original surface back to its original shape, responding
to any collisions that occur with the coarser layers that might occur during
inflation. From coarse to fine, each layer then fully encloses the next while
retaining a snug fit and respecting the original surface topology. We show how
these hierarchies are useful for fast PDE solving, low-frequency data
upsampling, and geometric modeling.

\alec{We might need to be careful about the term ``nested''. It seems in the
multigrid literature ``nested'', ``node-nested'', and ``non-nested'' have very
specific meanings. Might be that the best option is to use ``nested'', but
explicitly differentiate it with multigrid lit's use.}
\end{abstract}


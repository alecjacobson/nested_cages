\section{Introduction}
\label{sec:introduction}

\alec{need to stress generality: multires hierarchies, collision hierarchies,
cages for deformation, cages for computation}

As the complexity and size of objects we computationally model continue to
grow, acceleration algorithms for handling these large become
increasingly important. One powerful technique for managing this complexity is to decompose the
high-resolution mesh into a hierarchy of increasingly coarse approximations or
\emph{cages}: for example, Eulerian PDEs can be solved efficiently on a fine
mesh using multiresolution FEM techniques, which
first efficiently solve for a coarse approximate solution on the coarsest levels of the
hierarchy, and then polishing that solution on successively finer levels.
Coarse cages also find use in animation, where artist can specify large-scale deformations of
a character or object by adjusting a coarse cage; and in collision detection, where
a coarse cage can be used to efficiently but conservatively exclude pairs of
objects from having collided. For all these applications, the key to good
performance is the ability to generate a good-quality multiresolution
hierarchy.

\noindent \textbf{Desiderata} Although different applications place different
emphases on what makes for a desirable multiresolution hierarchy, an ideal hieraracy possess several properties useful across a broad spectrum of applications:
\begin{itemize}
\item Each mesh of the hierarchy completely \emph{encloses} the next-finest mesh (as well as all finer
levels of the hierarchy.) Most algorithms for transferring the
pose of a deformation cage to a detailed object only guarantee small distortion of the
object and lack of element-inversion artifacts if the object is entirely
contained within the cage, for instance. And if the cages are to be used to
accelerate collision detection, it is essential that the cage conservatively
enclose the object. Accurate handling of boundary conditions in multigrid FEM simulations often requires that the coarse meshes enclose the fine ones, particularly near Neumann boundaries~\cite{Chan96}.
\item All levels of the hierarchy respect the \emph{topology} of the finest mesh, and do not introduce
any large global errors in the object geometry. It is important to
differentiate between two types of distances when building the hierarchy:
\emph{embedded distance} in ambient space, and \emph{intrinsic distance}
between two points: the distance of the shortest path connecting the two points through the material. It is possible for two points that are nearby in
embedded distance to be very far apart in intrinsic distance: consider, for
instance, a model of a human leaning over and nearly touching her toes. The
fingertips and toes are nearby in ambient space but intrinsically are meters
apart. It is critical for many applications that the coarsening process does
not fuse together these intrisically-distant parts of an object. In the case of the human model, it
is unacceptable if trying to animate the model's arm causes the leg to move as
well. As another example, consider calculating the vibration modes of a broken
ring; if the coarsening process fuses the ring together, the low-energy modes computed
will be largely incorrect.
\item Each mesh in the hierarchy fits \emph{tightly} around the next-finest level of the hierarchy.
Formally, tightness can be quantified using a user-specified energy that
minimizes Hausdorff distance between the cage and the fine mesh, or minimizes
the volume enclosed between the two meshes, etc. A tight fit minimizes the geometric error at the object boundary incurred by simplifying the input mesh.
\end{itemize}
Existing popular approaches to building a coarse cage satisfy some, but not
all, of these three properties. For instance,
\begin{itemize}
\item Simple decimation of the original surface mesh yields coarse approximations that fit very
tightly and preserve a high amount of detail for a given coarsened resolution,
and can preserve the topology of the original mesh, but there is no guarantee
that the decimated mesh will enclose the original.
\item Voxellizing the object to a regular grid is guaranteed to produce an
enclosing cage, but the fit will not be tight and intrinsically distant
features can be fused together if the grid is too coarse. Using an adaptive
grid, octree, etc. can improve the fit and topological fidelity, but unless the
object contains mostly axis-aligned features the resulting cage will not
efficiently fit the object.
\item Level-set methods can find a well-fitting cage that encloses the object,
but can fuse spatially proximate features, changing the object topology.
\end{itemize}

\noindent\textbf{Our Contribution} We propose an algorithm for computing a
multiresolution hierarchy that satisfy all three of the desired properties. Given the
mesh $M_i$ for level $i$ of the hierarchy, we compute the next-coarsest level
$M_{i+1}$ using three steps (see figure XXX): first, we perform
topology-preserving decimation to get a coarse, but not enclosing, cage
$C_{i+1}$. We then shrink $M_i$, using a combination of surface-area- and
distance-minimizing flow (see section XXX) until $M_i$ is completely enclosed
by $C_{i+1}$. Finally, we rewind the flow, so that the vertices of $M_i$ move back towards
their original positions. During this reinflation, where the motion of $M_i$ is prescribed, we run a physical simulation to update the position of $C^{i+1}$. Contact forces
guarantee that $C_{i+1}$ remains an enclosing cage, and a fairness energy
maximizes the tightness of the fit. The final position of $C_{i+1}$ after this
simulation becomes the next cage $M_{i+1}$. We demonstrate this method on a wide variety of example meshes with challenging geometry and topology, for which existing methods are ill-suited (see section XXX). 

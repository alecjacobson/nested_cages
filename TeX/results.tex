
\section{Results and applications}
\label{sec:results}

% !TEX root = winding.tex
\begin{table*}
\centering
\ra{1.2}
\setlength{\tabcolsep}{5.5pt}
\rowcolors{2}{white}{lightbluishgrey}
\begin{tabular}{l r r r r r r r r r r r r r r r r r}
\rowcolor{white}
Model name  & \#F & Layers' \#F & Avg.\ Shrink & Avg.\ Re-inflate& Energy\\
\midrule
Anchor & 10,778 & 678 \ 1,068 \ 1,696 \ 2,694 \ 4,276 \ 6,788 & 8s & 34s & Volume \\
Armadillo & 12,000 & 470 \ 748 \ 1,190 \ 1,888 \ 2,998 \ 4,760 & 27s & 998s & Volumetric ARAP \\
Gargo & 13,500 & 531 \ 843 \ 1,339 \ 2,126 \ 3,375 \ 5,357 \ 8,504 & 2s & 90s & Volumetric ARAP\\
% Model9
Warrior & 26,658 & 1,048 \ 1,666 \ 2,644 \ 4,198 \ 6,664 \ 10,578 \ 16,790 & 37s & 497s & Volume  \\
Bunny & 34,832 & 1,371 \ 2,177 \ 3,455 \ 5,485 \ 8,708 \ 13,823 & 3s & 95s & Volumetric ARAP \\
Horse & 39,696 & 1,562 \ 2,480 \ 3,936 \ 6,250 \ 9,922 \ 15,752 \ 25,006 & 12s & 321s & Surface ARAP \\
Pelvis & 40,316 & 1,586 \ 2,516 \ 3,998 \ 6,346 \ 10,078 \ 15,994 \ 25,394 & 11s & 460s & Volume  \\
Octopus & 100,000 & 3,936 \ 6,248 \ 9,920 \ 15,748 \ 25,000 \ 39,684 \ 62,994 & 11s & 227s & Volume \\
\bottomrule
\end{tabular}
\caption{
``Avg.\ Shrink'' is the average time to shrink each layer inside the previous.
``Avg.\ Re-inflate'' is the average time to re-inflate each layer.
\alec{Let's make this a single column table and remove or abbreviate the
``Layers' \#F'' column.}
}
\label{tab:timings}
\end{table*}


\begin{figure}
  \includegraphics[width=\linewidth]{figs/dane-horse-vs-ben-chen}
  \caption{Methods like \protect\cite{Ben-Chen:2009:SDT} use coarse cages a
  computational workhorses to reduce complexity. Their iterative offset
  heuristic produces oversmoothed, loosely fitting cages.}
  \label{fig:dane-vs-ben-chen}
\end{figure}

\begin{figure}
  \includegraphics[width=\linewidth]{figs/swat-cage}
  \caption{The \emph{S.W.A.T. man} is an artifact-ridden mesh with 2806 pairs
  of intersecting triangles, 24 non-manifold edges, 51 boundary loops and 51
  connected components. Our shrinking flow to the input overlapping coarse cage
  is robust to such artifacts. Once inside, we re-inflate and produce a fully
  exterior cage.}
  \label{fig:swat-cage}
\end{figure}

\begin{figure}
  \includegraphics[width=\linewidth]{figs/warrior-poisson-vs-cgal}
  \caption{Contouring a shape's distance field requires aggressive spacing
  between isolevels to produce valid nesting. Semantically distant parts
  \emph{fuse} together, destroying the shape-awareness of the volumetric
  Laplacian.} 
  \label{fig:warrior-poisson}
\end{figure}

\begin{figure}
  \includegraphics[width=\linewidth]{figs/horse-25-layers}
  \caption{Our tight nesting property is robust even when the number of layers
  is large.}
  \label{fig:horse-25-layers}
\end{figure}

\begin{figure}
  \includegraphics[width=\linewidth]{figs/zoo}
  \caption{Each row shows left to right: input model, slice through all nested
  layers, and outermost, coarsest layer.}
  \label{fig:zoo}
\end{figure}

\subsection{Needs figures made:}
- Zoo:
  - anchor
  - arma
  - bunny
  - gargo
  - Model1
  - Model3
  - Model6
  - Model7
  - Model9
  - pelvis
  - octopus
  - horse
- couplingdown comparison of decimators
- rampant stress test with small number of layers vs CGAL
- cgal iso surface generation comparison vs CGAL
- Plain decimation creating bad overlaps
- kenshi's hand


- failure case for simpler flows: mean curvature flow on fine mesh does not
  go inside coarse mesh, may even intersect itself
    - pelvis or rampant failing to go inside at all
    - octopus flowing inside but shrinking too much and puts parts in wrong
      place
    - or 2d flow
- voxelize to get containing shell--> decimate --> run our pipeline to force
  outside: use for volumetric calculations: This has the benefit that the
  containing mesh is nice quality (cf. \cite{Jacobson:WN:2013}) and tightly
  fitting (cf. \cite{Xu:2014:SDF}).
- didactic 2D figures
  - flow
  - expansion
- rampant 100 layers
+ swat symmetry cage
+ Kenshi's hand: planarization energy + arap
- Kenshi's hand: compute HC and make cage-based deformation

- comparison of volumetric arap, surface arap, volume
- bunny: add random noise to initial model and run pipeline
  - rerun with higher resolution
- xyz-dragon High resolution stress test
+ alec: implement multigrid
  + Multigrid Poisson equation. Show that naive non-nested meshes will have
    poor/no convergence: alligator teeth, etc.
      - generic poisson equation in 2d and in 3d
        + compare to level set decimation
        + compare to qslim
      - geodesics in heat 
  - Multires automatic skinning weights (i.e. QP solve). See Baohua Wu's thesis.
  - Multi-res ARAP deformation/physical elastics
  - Bijective mapping (Locally injective mapping of tet mesh conforming to all
    layers): maintains nesting.
- beast for zoo

- cartoon heart
- left atrium
- flow Model3 (human) inside and inflate Model4 (armour)
- buddha using open flipper decimation
- propagate cages along animation sequence (just re-inflation)
- cage transfer: transfer horse cage to giraffe

\alec{discuss how we chose our default decimation ratio}

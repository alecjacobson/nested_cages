%%% The ``\documentclass'' command has one parameter, based on the kind of
%%% document you are preparing.
%%%
%%% [annual] - Technical paper accepted for presentation at the ACM SIGGRAPH 
%%%   or SIGGRAPH Asia annual conference.
%%% [sponsored] - Short or full-length technical paper accepted for 
%%%   presentation at an event sponsored by ACM SIGGRAPH
%%%   (but not the annual conference Technical Papers program).
%%% [abstract] - A one-page abstract of your accepted content
%%%   (Technical Sketches, Posters, Emerging Technologies, etc.). 
%%%   Content greater than one page in length should use the "[sponsored]"
%%%   parameter.
%%% [preprint] - A preprint version of your final content.
%%% [review] - A technical paper submitted for review. Includes line
%%%   numbers and anonymization of author and affiliation information.


\documentclass[annual]{acmsiggraph}

\usepackage{graphicx}
\usepackage[usenames,dvipsnames]{color}
\usepackage{amsmath}    % need for subequations
\usepackage{amssymb}    % need for things like varnothing
\usepackage{dsfont}
\usepackage{mathrsfs}
\usepackage{placeins}
\usepackage{microtype}
\usepackage{wrapfig}
\usepackage{soul}

% fix spacing before \paragraph
\usepackage{titlesec}
\titlespacing{\paragraph}{%
  0pt}{%              left margin
  0.2\baselineskip}{% space before (vertical)
  1em}%

%%% If you are submitting your paper to one of our annual conferences - the 
%%% ACM SIGGRAPH conference held in North America, or the SIGGRAPH Asia 
%%% conference held in Southeast Asia - there are several commands you should 
%%% consider using in the preparation of your document.

%%% 1. ``\TOGonlineID''
%%% When you submit your paper for review, please use the ``\TOGonlineID''
%%% command to include the online ID value assigned to your paper by the
%%% submission management system. Replace '45678' with the value you were
%%% assigned.

\TOGonlineid{0171}

%%% 2. ``\TOGvolume'' and ``\TOGnumber''
%%% If you are preparing a preprint of your accepted paper, and your paper
%%% will be published in an issue of the ACM ``Transactions on Graphics''
%%% journal, replace the ``0'' values in the commands below with the correct
%%% volume and number values for that issue - you'll get them before your
%%% final paper is due.

\TOGvolume{}
\TOGnumber{}

%%% 3. ``TOGarticleDOI''
%%% The ``TOGarticleDOI'' command accepts the DOI information provided to you
%%% during production, and which makes up the URLs which identifies the ACM
%%% article page and direct PDF link in the ACM Digital Library.
%%% Replace ``1111111.2222222'' with the values you are given.

\TOGarticleDOI{1111111.2222222}

%%% 4. ``\TOGprojectURL'', ``\TOGvideoURL'', ``\TOGdataURL'', ``\TOGcodeURL''
%%% If you would like to include links to personal repositories for auxiliary
%%% material related your research contribution, you may use one or more of
%%% these commands to define an appropriate URL. The ``\TOGlinkslist'' command
%%% found just before the first section of your document will add hyperlinked
%%% icons to your document, in addition to hyperlinked icons which point to
%%% the ACM Digital Library article page and the ACM Digital Library-held PDF.

%%% Replace ``PAPER TEMPLATE TITLE'' with the title of your paper or abstract.

\title{Nested Cages}

%%% The ``\author{}'' command takes the names and affiliations of each of the
%%% authors of your paper or abstract. The ``\thanks{}'' command takes the
%%% contact information for each author.
%%% For multiple authors, separate each author's information by the ``\and''
%%% command.

\author{
Leonardo Sacht$^1$ \quad
Etienne Vouga$^2$ \quad
Alec Jacobson$^{3}$ \quad
}
\affiliation{
$^1$Universidade Federal de Santa Catarina\quad
$^2$University of Texas at Austin \quad
$^3$Columbia University \quad
}

%%% The ``pdfauthor'' command accepts the authors of the work,
%%% comma-delimited, and adds this information to the PDF metadata.

\pdfauthor{Leonardo Sacht, Etienne Vouga, Alec Jacobson}

%%% Keywords that describe your work. The ``\keywordlist'' command will print
%%% them out.

\keywords{Mesh decimation, geometric flow, generalized barycentric coordinates, multigrid solver}

%%% The ``\begin{document}'' command is the start of the document.

%%% If you have user-defined macros, you may include them here.
% Comments
\newcommand{\leo}[1]{{\bf\textcolor[rgb]{0.9,0.0,0.0}{Leo: #1}}}
\newcommand{\todo}[1]{{\bf\textcolor{red}{TODO: #1}}}
\newcommand{\alec}[1]{{\bf\textcolor[rgb]{0.2,0.8,0.2}{Alec: #1}}}
\newcommand{\etienne}[1]{{\bf\textcolor[rgb]{0.9,0.5,0.0}{Ladi: #1}}}
\newcommand{\cheatvspace}[1]{}
\newcommand{\newhl}[1]{#1}
\newcommand{\newnewhl}[1]{#1}

%% To use cached, low resolution figures (for a smaller output pdf)
%% Run:
%%   mogrify -path lowresfigs/ -compress jpeg -density 150 fig/*.pdf
%% And uncomment this:
%\newcommand{\figs}{lowresfigs}
%\usepackage{background}
%\backgroundsetup{%
%  scale=1,       %% change accordingly
%  angle=0,       %% change accordingly
%  opacity=.2,    %% change accordingly
%  color =red,  %% change accordingly
%  contents={{\fontsize{50}{60}\selectfont LOW RES FIGURES}}
%}
% And comment this:
\newcommand{\figs}{figs}

\let\vc = \mathbf
\let\mat = \mathbf


\newcommand{\refequ}[1] {Equation~(\ref{equ:#1})} 
\newcommand{\reffig}[1] {Figure~\ref{fig:#1}} 
\newcommand{\refoutsidefig}[1] {Figure~#1} 
\newcommand{\reftab}[1] {Table~\ref{tab:#1}} 
\newcommand{\reflem}[1] {Lemma~\ref{lem:#1}} 
\newcommand{\refsec}[1] {Section~\ref{sec:#1}} 
\newcommand{\refapp}[1] {Appedix~\ref{app:#1}} 


% Notation

\newcommand{\R}{\mathds{R}}   % Real numbers
\newcommand{\n}{\vc{x}} 			% Coordinates of the nodes
\newcommand{\N}{\mat{X}} 			% Set of node positions
\newcommand{\s}{s} 			% length of bone incident on node i
\renewcommand{\S}{\mat{S}} 			% Set of lengths
\newcommand{\p}{\vc{p}} 			% Parent of a node
\newcommand{\vv}{\vc{v}} 			% Normalized edge vector
\newcommand{\RR}{\mat{R}} 		% Relative rotation at joint
\renewcommand{\u}{\vc{u}}			% User-provided positional constraints
\renewcommand{\a}{\vc{a}}     % Extra origins, for non spherical splitters
\newcommand{\user}{\text{user}} % subscript for user energy and weight
\newcommand{\lap}{\text{reg}} % subscript for laplacian energy and weight
\newcommand{\drag}{\text{drag}} % subscript for Inertia energy and weight
\newcommand{\q}{\vc{q}} % quaternion
\newcommand{\Q}{\mat{Q}} % absolute orientation of device node
\newcommand{\Qroot}{\Q_{0}} % Absolute orienation of device root
\newcommand{\W}{\mat{W}} % rotation matching device to existing bone
\newcommand{\hr}{\noindent\rule[0.5ex]{\linewidth}{1pt}}
\newcommand{\argmin}{\mathop{\text{argmin }}}
\newcommand{\ig}[2][]{\includegraphics[#1]{#2}}



% scientific notation
% http://tex.stackexchange.com/a/70532/13600
\usepackage{siunitx}
\sisetup{output-exponent-marker=\text{e}, bracket-negative-numbers,
open-bracket={\text{-}}, close-bracket={}}

\usepackage[mathletters]{ucs}
\usepackage[utf8x]{inputenc}
% μ
\usepackage{upgreek}
\usepackage{dblfloatfix}
\newcommand{\microsec}{\upmu{}\text{s}}

\newcommand{\transpose}{{\mathsf T}}



%  bold paragraph header without space after
\newcommand{\nospaceparagraph}[1]{{\sffamily\textbf{#1}}}
\renewcommand{\Re}{\operatorname{Re}}

\usepackage{enumerate}


% FIX LINE NUMBERS AFTER EQUATION
% http://changilkim.wordpress.com/2013/05/13/siggraph-latex-template-line-number-correction-hack/
\newcommand*\patchAmsMathEnvironmentForLineno[1]{%
\expandafter\let\csname old#1\expandafter\endcsname\csname #1\endcsname
\expandafter\let\csname oldend#1\expandafter\endcsname\csname end#1\endcsname
\renewenvironment{#1}%
{\linenomath\csname old#1\endcsname}%
{\csname oldend#1\endcsname\endlinenomath}}%
\newcommand*\patchBothAmsMathEnvironmentsForLineno[1]{%
\patchAmsMathEnvironmentForLineno{#1}%
\patchAmsMathEnvironmentForLineno{#1*}}%
\AtBeginDocument{%
\patchBothAmsMathEnvironmentsForLineno{equation}%
\patchBothAmsMathEnvironmentsForLineno{align}%
\patchBothAmsMathEnvironmentsForLineno{flalign}%
\patchBothAmsMathEnvironmentsForLineno{alignat}%
\patchBothAmsMathEnvironmentsForLineno{gather}%
\patchBothAmsMathEnvironmentsForLineno{multline}%
\patchBothAmsMathEnvironmentsForLineno{eqnarray}%
}

\begin{document}

%%% A ``teaser'' image appears under the title and affiliation information,
%%% horizontally centered, and above the two columns of text. This is OPTIONAL.
%%% If you choose to have a ``teaser'' image, it needs to be placed between
%%% ``\begin{document}'' and ``\maketitle.''

\teaser{
\vspace*{-0.5cm}
  \includegraphics[width=\textwidth]{\figs/bunny-shelf-teaser}
  \caption{
Given an input shape (yellow on bottom right), our method constructs \emph{nested} cages:
each subsequent mesh is coarser than the last and fully encloses it without
intersections. A slice through all layers (left), shows a tightly
encaged \emph{Bunny}.}
  \label{fig:teaser}  
}

%%% The ``\maketitle'' command must appear after ``\begin{document}'' and,
%%% if you have one, after the definition of your ``teaser'' image, and
%%% before the first ``\section'' command.

\maketitle

%%% Your paper's abstract goes in its own section.

\begin{abstract} 

Many tasks in geometry processing and physical simulation benefit from
multiresolution hierarchies. Ideal coarse domain approximations should satisfy
several desiderata for efficient and accurate multiresolution algorithms: the
coarsened domain should respect the original topology, \emph{enclose} the
original object, yet fit as tightly as possible.
%
Existing techniques such as surface mesh decimation, voxelization, or
contouring distance level sets, violate one or more of these desiderata.
%
We propose a solution that satisfies all three requirements by successively
constructing each next-coarsest level of the hierarchy, using a sequence of
decimation, flow, and contact-aware optimization steps.
%
From coarse to fine, each layer then fully encages the next while retaining a
snug fit and respecting the original surface topology.
%
We show that the method is applicable to a wide variety of shapes of complex
geometry and topology.
%
We demonstrate the effectiveness of our nested cages not only for
multiresolution solvers, but also for conservative collision detection, domain
discretization for elastic simulation, and cage-based geometric modeling. 

\end{abstract}




%%% ACM Computing Review (CR) categories.
%%% See <http://www.acm.org/class/1998/> for details.
%%% The ``\CRcat'' command takes four arguments.

%\begin{CRcatlist}
%  \CRcat{}{}{}{}
%\end{CRcatlist}
%
%%%% The ``\keywordlist'' command prints out the keywords.
%
%\keywordlist

%%% The ``\TOGlinkslist'' command will insert hyperlinked icon(s) to your
%%% paper. This includes, at a minimum, hyperlinked icons to the ACM article
%%% page and the ACM Digital Library-held PDF. If you added URLs to
%%% ``\TOGprojectURL'' or the other, similar commands, they will be added to
%%% the list of icons.
%%% Note: this functionality only works for annual-conference papers.

\TOGlinkslist

%%% The ``\copyrightspace'' command 
%%% Do not remove this command.

\copyrightspace

%%% This is the first section of the body of your paper.

%
%%textwidth: \printinunitsof{in}\prntlen{\textwidth}
%%%textwidth: 7.00137in
%%
%%linewidth: \printinunitsof{in}\prntlen{\linewidth}
%%%linewidth: 3.33461in
%%
%%\makeatletter
%%orig: \f@size pt \f@family 
%%%orig: 9pt ptm
%%\makeatother
%%\rmfamily

\section{Introduction}
\label{sec:introduction}

\alec{need to stress generality: multires hierarchies, collision hierarchies,
cages for deformation, cages for computation}

As the complexity and size of objects we computationally model continue to
grow, acceleration algorithms for handling these large become
increasingly important. One powerful technique for managing this complexity is to decompose the
high-resolution mesh into a hierarchy of increasingly coarse approximations or
\emph{cages}: for example, Eulerian PDEs can be solved efficiently on a fine
mesh using multiresolution FEM techniques, which
first efficiently solve for a coarse approximate solution on the coarsest levels of the
hierarchy, and then polishing that solution on successively finer levels.
Coarse cages also find use in animation, where artist can specify large-scale deformations of
a character or object by adjusting a coarse cage; and in collision detection, where
a coarse cage can be used to efficiently but conservatively exclude pairs of
objects from having collided. For all these applications, the key to good
performance is the ability to generate a good-quality multiresolution
hierarchy.

\noindent \textbf{Desiderata} Although different applications place different
emphases on what makes for a desirable multiresolution hierarchy, an ideal hieraracy possess several properties useful across a broad spectrum of applications:
\begin{itemize}
\item Each mesh of the hierarchy completely \emph{encloses} the next-finest mesh (as well as all finer
levels of the hierarchy.) Most algorithms for transferring the
pose of a deformation cage to a detailed object only guarantee small distortion of the
object and lack of element-inversion artifacts if the object is entirely
contained within the cage, for instance. And if the cages are to be used to
accelerate collision detection, it is essential that the cage conservatively
enclose the object. Accurate handling of boundary conditions in multigrid FEM simulations often requires that the coarse meshes enclose the fine ones, particularly near Neumann boundaries~\cite{Chan96}.
\item All levels of the hierarchy respect the \emph{topology} of the finest mesh, and do not introduce
any large global errors in the object geometry. It is important to
differentiate between two types of distances when building the hierarchy:
\emph{embedded distance} in ambient space, and \emph{intrinsic distance}
between two points: the distance of the shortest path connecting the two points through the material. It is possible for two points that are nearby in
embedded distance to be very far apart in intrinsic distance: consider, for
instance, a model of a human leaning over and nearly touching her toes. The
fingertips and toes are nearby in ambient space but intrinsically are meters
apart. It is critical for many applications that the coarsening process does
not fuse together these intrisically-distant parts of an object. In the case of the human model, it
is unacceptable if trying to animate the model's arm causes the leg to move as
well. As another example, consider calculating the vibration modes of a broken
ring; if the coarsening process fuses the ring together, the low-energy modes computed
will be largely incorrect.
\item Each mesh in the hierarchy fits \emph{tightly} around the next-finest level of the hierarchy.
Formally, tightness can be quantified using a user-specified energy that
minimizes Hausdorff distance between the cage and the fine mesh, or minimizes
the volume enclosed between the two meshes, etc. A tight fit minimizes the geometric error at the object boundary incurred by simplifying the input mesh.
\end{itemize}
Existing popular approaches to building a coarse cage satisfy some, but not
all, of these three properties. For instance,
\begin{itemize}
\item Simple decimation of the original surface mesh yields coarse approximations that fit very
tightly and preserve a high amount of detail for a given coarsened resolution,
and can preserve the topology of the original mesh, but there is no guarantee
that the decimated mesh will enclose the original.
\item Voxellizing the object to a regular grid is guaranteed to produce an
enclosing cage, but the fit will not be tight and intrinsically distant
features can be fused together if the grid is too coarse. Using an adaptive
grid, octree, etc. can improve the fit and topological fidelity, but unless the
object contains mostly axis-aligned features the resulting cage will not
efficiently fit the object.
\item Level-set methods can find a well-fitting cage that encloses the object,
but can fuse spatially proximate features, changing the object topology.
\end{itemize}

\noindent\textbf{Our Contribution} We propose an algorithm for computing a
multiresolution hierarchy that satisfy all three of the desired properties. Given the
mesh $M_i$ for level $i$ of the hierarchy, we compute the next-coarsest level
$M_{i+1}$ using three steps (see figure XXX): first, we perform
topology-preserving decimation to get a coarse, but not enclosing, cage
$C_{i+1}$. We then shrink $M_i$, using a combination of surface-area- and
distance-minimizing flow (see section XXX) until $M_i$ is completely enclosed
by $C_{i+1}$. Finally, we rewind the flow, so that the vertices of $M_i$ move back towards
their original positions. During this reinflation, where the motion of $M_i$ is prescribed, we run a physical simulation to update the position of $C^{i+1}$. Contact forces
guarantee that $C_{i+1}$ remains an enclosing cage, and a fairness energy
maximizes the tightness of the fit. The final position of $C_{i+1}$ after this
simulation becomes the next cage $M_{i+1}$. We demonstrate this method on a wide variety of example meshes with challenging geometry and topology, for which existing methods are ill-suited (see section XXX). 

\section{Related work}
\label{sec:related}


\section{Method}
\label{sec:method}

% establish input and output
The input to our method is a sequence of $k+1$ potentially overlapping
triangles meshes $(\V_0,\F_0),(\V_1,\F_1),\dots,(\V_k,\F_k)$, where $\V_i$ is a
list of 3D vertex positions and $\F_i$ is a list of triangle index triplets.
%
In general, we only assume that each mesh \emph{approximates} the surface of the same
solid object.
%
In a typical scenario, $(\V_0,\F_0)$ is a high-resolution \emph{original} mesh and
$(\V_1,\F_1),\dots,(\V_k,\F_k)$ are decimations of decreasing resolution.
%
We require that each input mesh is \emph{watertight} \cite{Dey:2003jf}:
%
free of open boundaries, non-manifold elements, or
self-intersections.\footnote{Our input assumptions are stricter than the
\emph{solidness} of \cite{Bernstein:2013:PHH}. That definition permits
non-manifold ``shared'' vertices and edges which confuse decimators.}
%
Depending on the application, troublesome input meshes may be \emph{cleaned} as
a pre-process using available tools (e.g.\
\cite{Attene:2010vv,Jacobson:WN:2013}).

The output of our method is a new sequence of $k$ lists of vertex positions 
$\V'_1,\dots,\V'_k$ such that all points on $(\V'_{i-1},\F_{i-1})$ are
strictly
\emph{inside} $(\V'_i,\F_i)$ for $i=1\dots k$, letting $\V'_0 = \V_0$ (see
\reffig{2d-nested-layers-notaion}).

We opt to alter only the vertex positions (\emph{geometric embedding}) of each
mesh, and not the number or connectivity of vertices ($\F_i$ are unchanged). Among
other benefits, this design decision ensures that the number of vertices in each
mesh is exactly maintained.

The \emph{nesting} property of the output meshes is easily \emph{verified} by testing
that the winding number of at least one vertex of $\V'_{i-1}$ with respect to
$(\V'_i,\F_i)$ is positive and no intersections exist between
$(\V'_{i-1},\F_{i-1})$ and $(\V'_i,\F_i)$. 

Now we describe a general method that guarantees this nesting property while
optimizing any problem specific energy (e.g.\ distance between meshes, mesh
volumes).

Our method operates recursively on two meshes of the sequence at a time: we
compute $\V'_i$ by considering only the solution to the previous level
$(\V'_{i-1},\F_{i-1})$ and initial mesh $(\V_i,\F_i)$. 
%
Without loss of generality, let us simplify our notation and consider computing
$\V'_1$ from $(\V'_0,\F_0)$ and $(\V_1,\F_1)$. We refer to the $0$th
mesh as the ``fine mesh'' and the $1$st mesh as the ``coarse mesh''.
%
Computing new coarse mesh vertex positions $\V'_1$ involves three phases:
\emph{flow} the fine mesh until fully inside the coarse mesh, \emph{re-inflate}
the fine mesh to its original embedding while \emph{pushing} the coarse mesh
out of the way, and finally \emph{optimize} the coarse mesh embedding (see
\reffig{2d-pipeline}).

\subsection{Flow}
\label{sec:flow}

Without loss of generality, the input fine mesh $(\V'_0,\F_0)$ and coarse mesh
$(\V_1,\F_1)$ are free of \emph{self}-intersections,
%
but, in general, the fine mesh will overlap the coarse mesh: some triangles of
the fine mesh will intersect those of the coarse mesh, implying that some
portion of $(\V'_0,\F_0)$ lies outside of $(\V_1,\F_1)$. Equivalently, there
exists a non-empty set of points on $(\V'_0,\F_0)$ with \emph{positive} signed
distance with respect to $(\V_1,\F_1)$ \footnote{Assuming a negative
inside, positive outside convention.}.
%
Our idea is to find a new embedding $\widehat{\V}_0$ that \emph{minimize
signed distance} integrated over all points $\p$ of the fine mesh to the
closest point $\q$ on the coarse mesh:
\begin{align}
\label{equ:sd-energy}
 Φ(\widehat{\V}_0) &= ∫_{(\widehat{\V}_0,\F_0)} s(\p) u(\p) \,dA,\\
& u(\p) = \left\| \p - \argmin_{\q \in (\V_1,\F_1)} \|\p-\q\|\right\|,\\
& s(\p) = \begin{cases}
  1 & \text{ if $\p$ is outside $(\V_1,\F_1)$},\\
  0 & \text{ if $\p$ is exactly on $(\V_1,\F_1)$},\\
  -1 & \text{ if $\p$ is inside $(\V_1,\F_1)$},
\end{cases}
\end{align}

where $u(\p)$ is the \emph{unsigned} distance from $\p$ to the coarse mesh and
$s(\p)$ modulates by the appropriate sign.

\leo{$dA$ has to be fixed, and we are using the one induced by the initial embedding.
If we don't fix, we don't flow all the way towards the medial axis. I have an analytic example
for a sphere, not sure how much details we should provide for this point.}

\leo{I would prefer to use $d$ for distance, instead of $u$}.

\alec{Are we actually using signed \emph{squared} distance?}

\leo{We are using the one whose gradient is the vector connecting to the closest 
point on the coarse mesh. I think it's not squared then.}

It is important that $\p$ consider \emph{all points} on the fine mesh, not just
vertices. Though \emph{all vertices} in $\widehat{\V}_0$ may lie inside
the coarse mesh, parts of edges and facets might still lie outside (see
\reffig{2d-cutting-corner}). It is also important that $\q$ consider \emph{any
point} on the coarse mesh, as the nearest vertex may be arbitrarily farther
away than the closest point along an edge or facet.

We may immediately write our energy as sum of integrals over each triangle 
$\{a,b,c\}$ in $(\widehat{\V}_0,\F_0)$:
\begin{equation}
Φ(\widehat{\V}_0) = ∑_{\forall \{a,b,c\}} ∫_{\p \in \{a,b,c\}} s(\p) u(\p)\,dA.
\end{equation}

We minimize the energy by taking small steps opposite the gradient direction
for each vertex position $\widehat{\vv}$ in $\widehat{\V}_0(t)$ as a function
of a hypothetical \emph{time} variable $t$:
\begin{equation}
\dd{\widehat{\vv}}{t} = -\Grad_{\widehat{\vv}} Φ(\widehat{\V}_0).
\end{equation}
%
By following this gradient, we \emph{flow} the fine mesh vertices, until
$\widehat{\V}_0(t)$ is fully inside the coarse mesh.

Our continuous energy in \refequ{sd-energy} is similar to the data terms found
in non-rigid registration techniques \alec{cite non-rigid ICP paper}. However,
the sign modulator $s(\p)$ makes an important difference. Minimizing unsigned
(positive) distances would encourage points toward the surface of the coarse
mesh. Instead, by encouraging negative distances, points are attracted to the
medial axis \emph{within} the coarse mesh.

To differentiate the 
continuous energy in \refequ{sd-energy}, we
%
first approximate the continuous integral using a discrete set of quadrature 
evaluation points $\p_i$ and corresponding weights $w_i$ for each triangle:
\begin{align}
Φ(\widehat{\V}_0) &\approx ∑_{\forall \{a,b,c\}} ∑_i w_i s(\p_i) u(\p_i),\\
& \p_i = 
\lambda_a \vv_a + 
\lambda_b \vv_b + 
\lambda_c \vv_c,
\end{align}
where $\lambda_a,\lambda_b,\lambda_c$ are the barycentric coordinates locating $\p_i$
in the triangle $\{a,b,c\}$.

\leo{If $w_i$ are per-triangle weights (area in the initial embedding), I think it would be more clear
to put it outside the second summation and replace by $w_{\{a,b,c\}}$ (since it corresponds
to the $\{a,b,c\}$ triangle, not quadrature point $p_i$).}

%
We use second-order quadrature rules and see diminishing returns with more
exact schemes.
%
However, the difficulty of differentiating $u(\p_i)$ remains. To tackle this,
we adapt the successful iterative closest point approach of non-rigid
registration techniques. Namely, we assume that the closest point $\q_i^*$ to each
$\p_i$ remains constant during each small time step. Likewise, we assume that
the sign $s(\p_i)=s_i^*$ remains constant during each time step.

Now, the we can derive a
gradient for the $j$th mesh vertex $\widehat{vv}_i$. 
Without loss of generality, if we assume triangle indices ${a,b,c}$ are always
\emph{rotated} so that $a=j$, if any, then 
\begin{align}
\dd{\widehat{\vv}_i}{t} &\approx -\Grad_{\widehat{\vv}_i} 
∑_{\forall \{a,b,c\} | a = j} ∑_i w_i s(\p_i) u(\p_i),\\
&= -\Grad_{\widehat{\vv}_i} 
∑_{\forall \{a,b,c\} | a = j} ∑_i w_i s_i^* \|\p_i -\q_i^*\|,\\
&= -
∑_{\forall \{a,b,c\} | a = j} ∑_i w_i s_i^* \Grad_{\widehat{\vv}_i} \|\p_i -\q_i^*\|,\\
&=-
∑_{\forall \{a,b,c\} | a = j} ∑_i w_i s_i^* \Grad_{\widehat{\vv}_i}
\|
\lambda_a \vv_i + 
\lambda_b \vv_b + 
\lambda_c \vv_c
-\q_i^*\|,
\end{align}
\alec{whhhhhhaaat. My gradient has run amok. How come I'm not getting that we
have a weighted sum of directions toward the closest points to each quadrature
point?}

\leo{It would be easier to call $s \cdot u$ as a single function, to avoid product rules.
Also, you are forgetting a chain rule (since each quadrature points depends on the embedding).
I have it all written down in a pptx, could share or simply write this derivation.}
\begin{align}
&=-
∑_{\forall \{a,b,c\} | a = j} ∑_i w_i s_i^* \g_i,\\
&\text{ where } \g_i = \begin{cases}
\frac{\p_i -\q_i^*}{\|\p_i -\q_i^*\|} & \text{ if } \|\p_i
-\q_i^*\|<\epsilon,\\
\n(\q_i^*) & \text{ otherwise },
\end{cases}
\end{align}
where $\n(\q_i^*)$ is the unit normal at $\q_i^*$. This normal is chosen with
care. To ensure that it points \emph{inside} the coarse mesh, we determine if
$\q_i^*$ lies near a vertex, near an edge or in the middle of a triangle and
use an interior angle weighted average of incident triangle normals, a uniform
average of incident triangle normals or triangle normal,
respectively \cite{Baerentzen:2005:SDC}.

We iterate this flow stepping with magnitude proportional to a small ``time
step'': $∆t \approx XXXXXX$ \alec{what do we use?} \leo{Default is $10^{-3}$,
which works well for models scaled between -1 and 1. Can be tuned as optional parameter}. 
After each time step we
recompute signs $s_i^*$ and closest points $\q_i^*$ all quadrature points. We
terminate if all signs are negative \emph{and} no intersecting faces are found
between $(\widehat{\V}_0,\F_0)$ and $(\V_1,\F_1)$.

There is no formal, provable guarantee that this flow will succeed. Indeed, in
rare, difficult cases the flow converges without moving the fine mesh fully
inside the coarse mesh. This may occur if the medial axis of the coarse mesh
lies outside of the fine mesh. This is especially rare if the input sequence
results from decimation as decimation effectively \emph{prunes} and simplifies
a mesh's medial axis.

Rather than admit defeat, we propose a additional step which alleviates many
occurrences of this problem. We reverse the picture and expand the coarse mesh,
flowing it away from the current fine mesh along its signed distance field.

\leo{I think it is too early to put this last paragraph here. I'd rather discuss it in a limitation section.
Could be a bit harming here. It would be nicer to show working results here.}

\alec{Might want to answer question: why not just expand the coarse mesh or
always do both?}

\leo{Expansion is more complicated because it is performed in a coarser mesh. Additionally,
we have to run a physical simulation at every step to guarantee it does not tangle, which 
is not needed for the fine mesh.}

\alec{Relationship to David's penetration depth energy: needs intersection free
state; and Untangling Cloths' volume minimization: might push coarse mesh
\emph{inside} fine mesh instead.}

\alec{Should state that we 1) don't care if $(\widehat{\V}_0,\F_0)$ becomes
self-intersection or 2) cannot become self-intersecting.}

\leo{We don't care.}

\subsection{Re-inflation}
\label{sec:reinflation}

At this point, we have flowed the fine mesh vertices so that
$(\widehat{\V}_0,\F_0)$ is fully inside $(\V_1,\F_1)$. We now restore the fine
mesh to its original vertex positions $\V_0$, detecting and resolving collisions with
the coarse mesh along the way.

To do this, we iterate through the vertex positions computed for each time step
during the flow described in \refsec{flow} \emph{in reverse}.
%
Doing so without moving the coarse mesh vertices would immediately result in
fine-coarse intersections.

We can describe each reverse step in our flow in terms of a displacement per
time step, that is, in terms of \emph{velocities}. For the fine mesh, the
positions after the next reverse time step are known, and thus so are its
velocities:
\begin{align}
\widehat{\V}_0(t-∆t) &= 
\widehat{\V}_0(t) - ∆t \widehat{\U}_0(t),\\
\widehat{\U}_0(t) &= 
\frac{\widehat{\V}_0(t) - \widehat{\V}_0(t-∆t)}{∆t}
\end{align}
where $\widehat{\V}_0(t)$ are the vertex positions of the fine mesh at ``flow
time'' $t$ with the positions returning to their input positions at time zero
$\widehat{\V}_0(0) = \V'(0)$, and $\widehat{\U}_0(t)$ are the instantaneous
per-vertex 3D velocity vectors at time $t$.
%
\alec{The signs are funny since we're flowing backwards.}

For the coarse mesh, the positions $\bar{\V}_1(t-∆t)$---and in turn velocities
$\U_1(t)$---are not fixed. In general, there are an infinite number of
\emph{feasible} choices of $\U_1(t)$ so that the repositioned coarse mesh
$(\bar{\V}_1(t-∆t),\F_1)$ remains free of intersections with itself and with the
\emph{re-inflating} fine mesh $(\widehat{V}_0(t-∆t),\F_0)$.
%
To regularize this problem, we minimize the change in position, or equivalently
minimize the magnitude of velocities.
%
Our reverse time step problem becomes:
%
\begin{align}
\label{equ:simulation}
&\argmin_{\U_1(t)} \left\|\U_1(t)\right\|^2,\\
&\text{ subject to: }\\
&(\bar{\V}_1(t-∆t),\F_1) \text{ does not intersect itself},\\
&(\bar{\V}_1(t-∆t),\F_1) \text{ does not intersect } (\V_0(t-∆t),\F_0).
\end{align}

\alec{This is just disallowing instantaneous collisions. We actually model a
harder problem: no continuous collisions. ``Why?''}

By reformulating our problem in a manner familiar to physically based
simulation, we may leverage state of the art contact detection and response
methods.
%
Abstractly, we can treat these methods as a ``black box'', providing it
the fine mesh $(\V_0(t),\F_0)$ and coarse mesh $(\bar{\V}_1(t),\F_1)$ at the end of the
reverse flow time step and the desired velocities $\U_0(t)$ and $\U_1(t)$. 


There remains one interesting twist. Our problem requires the fine mesh to
return exactly to its original positions. In physically-based simulation
parlance, this is tantamount to assigning the fine mesh \emph{infinite mass}.
%
Many methods are fundamentally incapable of handling such constraints
\alec{cite synchronous penalty-based methods?}.
%
Instead, we experimented with two methods.

First, we adapt the ``velocity adjuster'', \alec{there's a better word than
``adjuster''} surface tracking method of \cite{Brochu:2009} to deal with
infinite masses. \alec{What did we have to change in El Topo? We decrease the
time step and retry assuming linear motion?}

\leo{We removed a phase called RIZ (Rigid Impact Zones), that allowed infinite mass 
vertices to not reach final positions. We also replaced Cholmod by IGL's min quad with
fixed (it better handled difficult cases). Finally, we also decrease time step as you described.}

In difficult cases, too many time step subdivisions are needed, suggesting
failure to make progress in finite time. To handle these hard cases, we fall
back on a second, more robust but slower method: speculative asynchronous
contact mechanics \cite{Ainsley:2012:SPA}. This method is an extension of the
only known method to guarantee intersection prevention and positive progress
for \emph{finite mass} objects \cite{Harmon:2009}. Our infinite masses pose an
interesting stress test for this method, but we see success, albeit at a slower
pace.

\subsection{Optimization}

So far we have allowed the coarse mesh positions to \emph{drift} as they are
\emph{pushed} outward by the re-inflating fine mesh. Afterwards, the embedding
of the coarse mesh $(\bar{\V}_1(0),\F_1)$ is strictly outside the fine mesh,
but may have strayed from an \emph{optimal} embedding.

Our final phase improves the positioning of the coarse mesh. The formulation is
general and allows the user to define the optimality metric for the
domain-specific problem at hand. 

Given a differentiable energy $E(\V'_1)$ and a current, feasible embedding
$(\V'_1,\F_1)$ we set up the following ``simulation'', reminiscent of
\refequ{simulation}:
\begin{align}
&\argmin_{\U'_1} \left\|\U'_1 - δ \Grad E(\V'_1) \right\|^2,\\
&\text{ subject to: }\\
&(\V'_1+\U'_1,\F_1) \text{ does not intersect itself},\\
&(\V'_1+\U'_1,\F_1) \text{ does not intersect } (\V'_0,\F_0).
\end{align}
where $δ \approx XXXXXX$\alec{what do we use?} \leo{Default is starting with
$δ = 10^{-1}$ and do a binary search to decrease energy while $δ > 10^{-3}$} is a small step size
parameter and $\Grad E(\V'_1)$ is the gradient of the user supplied energy with
respect to vertex positions. We solve this problem iteratively, setting $\V'_1
\leftarrow \V'_1 + \U'_1$ after each step until convergence using the same
black box solvers as in \refsec{reinflation}.

We experimented with a few different choices of optimality measures.
One natural choice is to minimize the coarse layer's volume, preferring a
tighter fit. In this case, we define:
\begin{align}
E_\text{volume}(\V'_1) &= ∫_{(\V'_1,\F_1)} 1 \,dA,\\
\Grad E_\text{volume}(\V'_1) &= \N_1,
\end{align}
where $\N_1$ are the per-vertex, area-weighted normals \alec{Should just be a
citation for this}.

\alec{what other energies? Dirichlet? ARAP? symmetry? planarization? Do we
really need to include definitions and gradients for all of these?}

\leo{Symmetry, planarization and proximity are partially implemented, could be done.
If we are going to have a supplemental PDF for other reasons, I would then add 
derivations there.}

\alec{Need to address: ``why not optimize while reinflating? why optimize only
at the end?'' We prioritize feasibility and performance before
quality...hmmm...that doesn't sound good.}

\leo{Because we are not concerned about the quality of intermediate steps.
We only care about the quality of the cage itself. And indeed it is faster.}

\alec{Need to address: ``why not try avoid collisions during decimation?''}

\alec{Might want to address: ``why accept pre-decimated meshes? Why not include
decimation as part of pipeline''?}

\leo{Default is to decimate while layers are generated. I.e., when we finish one layer, 
this layer is decimated to be the coarse layer for the next round. Alternatively,
all coarse layers may be prescribed as input.}

\alec{Need to address: ``why not shrink wrap a mesh from well outside?''}

\leo{The way we are doing now starts the final minimization at something close to what
we want, so less risk of finding local minima.}

\alec{We could handle energies which measure across layers. Minimize squared
separation distance between adjacent layers: might pay to expand middle layer
to balance space between fine and coarse layers. We can model this as one large
optimization problem, and recommend that one still proceed with a block
coordinate descent freezing all but one layer iteratively.}


\section{Results and applications}
\label{sec:results}


\section{Limitations and future work}
\label{sec:conclusion}
%
We plan to optimize the performance of both the signed distance field flow and
the re-inflation steps.
%
Collision detection and response dominates running time, and our prototype
naively recomputes acceleration data-structures rather than updating them
continuously.

Our insight to break the multi-layer nesting problem into pairwise subproblems
ensures tractability, but in some cases leads to converging at an
``artificial local minimum.'' If a coarse cage collides with itself during
inflation then it may create a pinch that blocks inflation of subsequent coarse
layers (see \reffig{homer}).
%
One solution is to iterate through the fine layers to make sufficient room in
these problem areas, but defining this relaxation direction is not obvious.

\begin{figure}
  \includegraphics[width=\linewidth]{figs/homer-fail}
  \caption{If a expanding coarse mesh collides with itself (green), it creates
  a \emph{pinch} preventing processing of further coarser layers.}
  \label{fig:homer}
\end{figure}

In cases where the input coarse cage begins too far away from the fine mesh,
the signed-distance flow will fail: for example, by flowing vertices of a
triangle into opposite parts of the coarse mesh. 
%
Adding a small amount of smoothing on the flowing fine mesh or expanding the
coarse mesh alleviates some of these problems, but a general solution is
illusive.
%
The correct assignment seems related to correctly matching medial axes of both
meshes: perhaps an avenue of future improvement.

\newhl{We believe the performance of our multigrid solver could be further
improved} by parameter tuning and experimenting with different coarsening
gradations. We would also like to consider using our meshes to build
multigrid proconditioners for conjugate gradient solvers.
%
We expect that higher order PDEs with more involved boundary conditions will
receive an even greater benefit from our nested cages.

It would be interesting to analyze formally the convergence of our nested cages
along the lines of \cite{chan1996convergence} who consider the then-available
non-nested hierarchies.

\newhl{Our cages are designed for \emph{volumetric} multigrid solvers, and are
not} immediately applicable to \emph{surface-based} multiresolution problems
(cf.\ \cite{Aksoylu2005msu,Chuang:2009:ELO}). Whether nesting is at all useful
for surface problems remains an open question.

In conclusion, nested cages prove to be a powerful tool in a variety of
applications. 
%
Our signed-distance flow consistently finds initial feasible states for our
constraint-based optimization.
%
By leveraging state-of-the-art collision handling tools from physically based
simulation, we are able to generate cages that in turn enable faster physical
simulations, more-efficient linear system solvers and better real-time
deformation user interfaces.
%
We hope that our algorithm's success encourages more multiresolution volumetric
methods using unstructured meshes in geometry processing, computer graphics,
and beyond.
%
We are also excited about possible practical solutions and theoretical
questions yet to be imagined with nested cages.

\begin{acks}
- Daniele, Derek, Eitan, Keenan, Ilya (meshes)
- the NSF grant DMS-1304211
- Eric Price, brainstorming the NP completeness proof
- Henrique, Sarah Abraham, proofreading
- The Columbia Computer Graphics Group is supported by the NSF, Intel, The Walt
  Disney Company, and Autodesk.
- Peter Schroeder, Richard Kenyon, Alexander I. Bobenko, Helmut Pottmann, and
  Johannes Wallner for organizing the DDG Oberwolfach and Seggau Geometry
  workshops.
\end{acks}


\begin{figure*}
  \includegraphics[width=\linewidth]{figs/zoo}
  \caption{Each triplet shows: input model, slice through all nested layers,
  and outermost, coarsest layer. We fit 50 layers tightly around \emph{Max
  Planck}'s head (top left). 
  %
  As a stress test, we purposefully continue nesting cages around \emph{Gargoyle} to a
  very coarse level (bottom left).
  %
  The topological holes of the high-genus \emph{Fertility} are maintained
  across all layers (top right). The deep concavity of the \emph{Mug} does not
  get smoothed away in coarser levels (bottom right).}
  \label{fig:zoo}
\end{figure*}

\bibliographystyle{acmsiggraph}
\bibliography{references}

\section*{Appendix: Flow counterexample}
%
\label{app:counterexample}
%
There exist polyhedra that cannot be continuously flowed inside themselves~\cite{Mathoverflow:206750}.
%
Consider the surface constructed in \reffig{vouga-fan}, right. Though inward
pointing normals can be defined at all vertices, any non-negative (even
infinitesimal) inward movement of the vertices will introduce ``edge-edge''
collisions with the original surface.
%
To see this, consider the set of points inside the polyhedron that can be "seen" by the central vertex $c$ of the star. This set changes discontinuously for any motion of $c$: no matter which direction is chosen to flow $c$, there exists some neighbor $n$ such that $c$ loses sight of not only $n$, but also an entire neighborhood around $n$. Therefore there exists no inward flow of all of the vertices where $c$ maintains sight with all of its neighbors.
\begin{figure}[hb]
\includegraphics[width=\linewidth]{figs/vouga-fan}
\caption{Take every other vertex around the star surface on the left and
rotate it about the origin to create the ``origami pinwheel'' on the right.
Vertices of this new surface cannot be continuously flowed inward without
creating intersections with the original surface.}
\label{fig:vouga-fan}
\end{figure}


\section*{Appendix B: Pseudocode}
%
\label{app:pseudocode}
\vspace{-5mm}

\begin{algorithm}
  \caption{$\text{Shrink}(\Cinp,\Fout) \rightarrow H$}
  \KwData{\\
    \begin{tabular}{ll}
      $\Cinp$ &  Initial coarse mesh (will remain constant) \\
      $\Fout$ &  Initial fine mesh (possibly overlapping with $\Cinp$) \\
    \end{tabular}
  }
  \KwResult
  {\\
    \begin{tabular}{ll}
      $H$  & Fine mesh history:
      $H \leftarrow \{\FVflow(0),\FVflow(∆t),\dots,\FVflow(N∆t)\}$
    \end{tabular}
  }
  \Begin{
    $H ← \{ F \}$ \comment{Initialize history with input fine mesh} \\
    \While{\upshape{$\Cinp$ does not nest $H$\text{.last}}}
    {
    $\nabla Φ  ← 0$ \comment{Initialize gradients to zero} \\
    \For{\upshape{each quadrature point $\pflow_i $ on $H$.last}}
    {
      $\q ← \text{closest point to } \pflow_i \text{ on }\Cinp$ \\
      $s ← \left\{
        \begin{array}{rl} 
           1 & \text{ if }\pflow_i\text{ is outside }\Cinp \\
          -1 & \text{otherwise} 
        \end{array} 
      \right.$ \comment{distance sign} \\
	 \uIf {$\| \pflow_i - \q \| > 1\mbox{\upshape{e}-}5$}
	 {
          $\g ← s (\pflow_i-\q)/\|\pflow_i - \q\|$ \\
	 }
	 \Else{
           \longcomment{if too close use normal \cite{Baerentzen:2005:SDC}}\\
           $\g ← \n(\q)$\\
         }
         \For{\upshape{each vertex $j$ in $H$.last}}
         {
            \longcomment{$w_i$ is the weight of the quadrature point and
            $\lambda_{ij}$ is the hat function of vertex $j$ evaluated at
            quadrature point $\pflow_i$}\\
            $\nabla Φ (j) ← Φ(j) +  w_i \lambda_{ij} \g$\\
         }        
         
    }
    $∆t ← 1\mbox{\text{e}-}3$ \comment{Default time step} \\
    $ H \text{.push}( H\text{.last} - ∆t \cdot \nabla Φ$  )
    }
  }
  \label{alg:shrink}
\end{algorithm}
\vspace*{-5mm}

\begin{algorithm}
  \caption{$\text{Reinflate}(H,\Cinp,\text{Energy}) \rightarrow C$}
  \KwData{\\
    \begin{tabular}{ll}
      $H$  & Fine mesh history:
      $H \leftarrow \{\FVflow(0),\FVflow(∆t),\dots,\FVflow(N∆t)\}$ \\
      $\Cinp$ &  Initial coarse mesh \\
      Energy &  Function object for re-inflation energy and gradient
    \end{tabular}
  }
  \KwResult
  {\\
    \begin{tabular}{ll}
      $C$ & Final nested cage \\
    \end{tabular}
  }
  \Begin{
    $F_0 ← H$.first \comment{initial fine mesh}\\
    $F ← H$.pop \comment{shrunken fine mesh}\\
    $C ← \Cinp$ \comment{initialize output cage}\\
    \While{\upshape{$H$ is not empty}}
    {
    $β ← β_{\text{init}} ( \ = 1\mbox{\text{e}-}2)$ \comment{Initial step size} \\
    $E_{\text{min}} ← \infty$ \comment{Initial minimum energy} \\
    $\U_F ← H\text{.pop} - F $ \comment{Desired velocities for fine mesh} \\
    \Repeat{
      $\U_C ← - β \ \nabla \text{Energy} (F_0,C,\Cinp)$
       \comment{\ditto{} coarse mesh} \\
       \longcomment{Filter velocities, e.g.\ modified \cite{Brochu:2009} or
        \cite{Ainsley:2012:SPA}}\\
       $\U_C ← \text{Filter}(F,\U_F,C,\U_C)$\\
	 \uIf {\upshape{Energy}$(C+\U_C)>E_{\text{min}}$}
	 {
	 $β ← β/2$ \comment{Decrease step size} \\
	 \lIf{$β < β_{\text{min}} ( \ = 1\mbox{\upshape{e}-}3)$} 
	 {
	    \textbf{break} \comment{No progress}
	 }
	}
	 \Else{
		$C ← C +\U_C$ \comment{Step coarse mesh} \\
         	$E_{\text{min}} ← \text{Energy}(C)$ \comment{Update energy}\\
		$β ← 1.1 β$ \comment{Slightly increase step size}
         }  
         \lIf{$\| \U_C \| < ∆C (=1\mbox{\upshape{e}-}5)$}
	 {
	    \textbf{break} \comment{"Converged"} 
	 }
    }
    $F ← F + \U_F$ \comment{Step fine mesh}

    }
  }
  \label{alg:reinflate}
\end{algorithm}


\end{document}

\section{Related work}
\label{sec:related}

Decimation

Non-nested decimation \cite{Garland:1997:SSU} and progressive meshes
\cite{Hoppe:1996:PM} \cite{Melax98} Typically the goal is appearance preserving
\cite{Cohen:1998:AS}, though later considered more elaborate tasks like
preserving sensations (haptics) \cite{Otaduy:2003:SPS}
\alec{I guess we could try to make the case for haptics scenarios where one-sided safe
contacts are ``essential''}

Combine decimation hierarchy with bounding volumes for faster, more accurate
collision detection
\cite{Otaduy:2003:CDH} 

Planar decimation avoiding flips (and therefor also self-intersections) in
output \cite{AnderssonGL09}.  \cite{gumhold2003intersection} new coarse mesh
won't contain self-intersections, but not trying to avoid intersections with
original fine mesh.

Rather than work against the mountain of advances in shape decimation we
complement them. Our method takes as input an arbitrary decimation and works as
a post process to nest each layer.

Cage generation

\cite{Ben-Chen:2009:SDT} creates an offset surface via Poisson reconstruction,

\cite{Deng:2011vr,Xian:2012tv}

Offset surfaces 

\cite{Campen:2010} Exact offset surface of triangle mesh is a curved surface
and only approximated by a fine mesh.  Could generalized 2d offset method
\cite{chen2005polygon} to 3D \cite{Jacobson:WN:2013}.  Or just mesh signed
distance field directions (e.g.\ using \cite{cgal}). Can be made robust to
nasty input with \cite{Xu:2014:SDF} and used for creating low res tet mesh for
simulation.
%
But offsets are bad idea for creating coarse mesh: too fine OR lose nesting
guarantee OR too far away (and start to combine features)

\cite{Shen:2004:IAI}, as an application, iteratively refine a MLS surface to
enclose an existing model: search for global parameter to minimize distance
indirectly. Presented algorithm determines if level-set is outside, but
contouring could still "slice" into original surface.  Further, topological
control is lost and close features are still quickly merged. 

Multi-grid 


\cite{McAdams:2011} Finite difference multigrid on voxelized input. Why not
voxelize? Glue geodesically distant parts together: voxel size determined by
min feature size and min void size. What about duplicated cells? Is there a
citation for this? Boundary handling overwhelms advantages of regular grid. Is
this even always well-defined? Nice if we had a counterexample.
\cite{Sykora09} replicates lattice cells in 2d, not sure how this affects his
shape-matching-style ARAP optimization.

\cite{wojtan2011liquid} fluid sims often couple Eulerian (finite difference)
grids with tracked surfaces. Seems that Wojtan et.\ al will identify grid cells
containing more than one distant surfaces and perform \emph{surgery on the
surface} rather than duplicate grid cells.

\cite{Adams:1999:PMS} Ritz-Galerkin multigrid, actually this is a mesh
decimation paper for creating ``node-nested'' tet meshes, building on
\cite{guillard1993}. Select subset of nodes, try to connect in a reasonable,
Delaunay-ish way. Simple linear restriction operator.

\cite{Brune:2011} Try to maintain corners while removing vertices to get a
node-nested coarsening.  Re-discretization (FEM-style) solves.

\cite{fish1995efficient} just an example of a (nonlinear) multigrid solver
which expects the user to provide (nested?) tetrahedral meshes as input.

\cite{feng1997non} more complicated restriction operator for Galerkin multigrid
on non-nested meshes, though it seems it's expected that the surfaces of each
mesh coincide: user provides meshes. I'm not sure how much better this
restriction would really be compared to simple barycentric one.

\cite{Debunne:2001:DRD} 
Adaptive simulations for visco-elastic solids. Computes a hierarchy of
non-nested decimations (QSlim \cite{Garland:1997:SSU}), tethers fine mesh
points to \emph{closest} coarse vertex.

\cite{CGCDP:2002} assume a subdivision lattice containing the deforming shape
as part of the input. Presumably this cage is modeled manually.

Flows
\cite{Kazhdan2012} remove singularities, also self-intersections
\cite{Sacht:SIV:2013}, in general, will not move inside original surface

Alternating global scaling and attraction, attempts to move inside. No
guarantees, requires parameter tuning \cite{Wang:2008}

\cite{Tagliasacchi:2012:MCS} coerces flow toward the medial axis

Alternatives to geometric multigrid: algebraic multi-grid (cite Brandt) or
algebraic techniques for special problem \cite{Krishnan:2013:EPL} (doesn't
generalize beyond ``M-matrices'')


\alec{Look in ETH MS Thesis for further citations. Ask Daniele and Derek
whether we should cite MS Thesis and how.}

\alec{Relationship to David's penetration depth energy: needs intersection free
state; and Untangling Cloths' volume minimization: might push coarse mesh
\emph{inside} fine mesh instead.}

\alec{Need to address: ``why not try avoid collisions during decimation?''}
